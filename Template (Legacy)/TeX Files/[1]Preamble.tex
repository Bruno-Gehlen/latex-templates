%   Preamble Contents:
%       - Packages    [Line 7]
%       - Commands    [Line 444]
%       - Document    [Line 571]
%       - Tips        [Line 623]

%-----------|--------------------------|-----------%
%           |                          |           %
%-----------|         PACKAGES         |-----------%
%           |__________________________|           %
%---%%%%%%%%%%%%%---% PACKAGES %---%%%%%%%%%%%%%---%

%===============================================
%---Input---

\usepackage[french,main=english]{babel} 
%   Hyphenation and typographical rules for a language

\usepackage[utf8]{inputenc} 
%   Input encoding

\usepackage[TS1,T1]{fontenc} 
%   Fonts encoding

\usepackage[backend=biber,style=numeric,sorting=none]{biblatex}
\addbibresource{Extras/References.bib}
%   Citations and references

\usepackage{xcolor}
%   Colors for words and boxes
\definecolor{corUrl}{RGB}{135,99,219}   %\
\definecolor{corLinks}{RGB}{135,99,219} % > Pretty 
\definecolor{corCite}{RGB}{135,99,219}  %/
\definecolor{DarkBlue}{RGB}{47,47,79}%{0,34,102}
\definecolor{ultramarine}{RGB}{0,32,96}
\definecolor{brownb}{RGB}{153,51,0}
\definecolor{corIFUSP}{RGB}{15,102,136}

\definecolor{ifBlue}{RGB}{15,102,136}%{234,83,70} not so red lol	 headers and hyperlinks
\definecolor{ifDarkBlue}{RGB}{91,27,22}%{91,27,22}   % text color in console (draft mode)
\definecolor{ifDarkerBlue}{RGB}{6,8,21}%{51,8,6}   % background color in console
\definecolor{fgLightRed}{RGB}{184,213,228}%{241,167,159}      % table header rows
\definecolor{fgVeryLightRed}{RGB}{233,247,254}%{248,226,224}  % table normal rows
\definecolor{fgDeepRed}{RGB}{238,42,36}
%
\definecolor{fgYellow}{RGB}{250,165,25}
\definecolor{fgLightYellow}{RGB}{229,208,66}
\definecolor{fgVeryLightYellow}{RGB}{255,237,163}
\definecolor{fgDarkYellow}{RGB}{99,74,33}
% others
\definecolor{bgLightGray}{RGB}{235,235,235}     % pseudocode background
%
\definecolor{LightGrey}{rgb}{0.96,0.96,0.96}    % title page
\definecolor{DarkGrey}{rgb}{0.83,0.83,0.83}
\definecolor{BaseColor}{rgb}{0.64,0.69,0.31}
\definecolor{fondtitre}{RGB}{0,66,92}



%===============================================
%---Figures---

\usepackage{graphicx} 
%   Manipulate words

\usepackage{subfigure} 
%   Use small figures inside a larger one

\usepackage{epstopdf}\epstopdfsetup{update} 
%   Convert .eps to .pdf figures

\usepackage{wrapfig} 
%   Figures besides text

\usepackage{float} 
%   H command for figures - put the figure where it appears

\usepackage{caption}
%   Custom captions



%===============================================
%---Math---

\usepackage{amsfonts} 
%   Mathematical fonts

\usepackage{tensor}
%   Tensors

\usepackage{amsmath} 
%   Mathematical structure
\numberwithin{equation}{subsection}

\usepackage{amssymb} 
%   Mathematical symbols

\usepackage{amstext} 
%   Mathematical \text command

\usepackage{slashed} 
%   Feynman slashed notation

\usepackage{dsfont}
%   Doublestroke Font

\usepackage{amsthm} 
%   Typesetting theorems

\usepackage{mathrsfs} 
%   RSFS fonts

\usepackage{booktabs} 
%   Enhance the tables

\usepackage{longtable} 
%   Allow write long tables

\newcommand\bmmax{2} 
%   New command to avoid several alphabets in mathematical mode

\usepackage{bm} 
%   To put bold and italic text in math mode

\usepackage{gensymb} 
%   Generic symbols, as degree, Celsius, per-thousand, etc

\usepackage{textcomp}
%   More symbols...

%\usepackage{systeme}
%   Create Linear Systems

%\usepackage{siunitx}
%   International System of Units (SI)

\usepackage{physics}
%   Physics notation

\usepackage{mathtools}
%   Math tools

\usepackage{tikz}
%   Draw math elements (graphics, etc)   

\usepackage[RPvoltages]{circuitikz}
%   Draw electric circuits



%===============================================
%---Formatting---

\usepackage[a4paper, top=2cm, inner=2cm, outer=2cm, bottom=2cm, includeheadfoot]{geometry} 
%   Document dimensions

%\usepackage{bibunits} 
%   Multiple bibliographies

\usepackage{setspace}\onehalfspacing 
%   space between lines

\usepackage{makeidx} 
%   creation of indexes

\usepackage{imakeidx}

\makeindex[columns=3, title=Alphabetical Index, intoc]

\usepackage{pdfpages} 
%   include pdf files

%\usepackage{mcite} 
%   multiple items in a single citation

\usepackage[Conny]{fncychap} 
%   Chapter headings styles
%   Options: Sonny, Lenny, Glenn, Conny, Rejne, Bjarne, Bjornstrup

\usepackage{layout} 
%   Layout of the document
  
\usepackage{url} 
%   Verbatim for urls

\usepackage{fancyvrb}
%   Fancy verbatim

\usepackage{multicol} 
%   Single and multiple columns

\usepackage{enumerate} 
%   Redefine labels of enumerate

\usepackage{indentfirst} 
%   Indent first paragraph after section

\usepackage{enumitem} 
%   Control layout of enumerate

\usepackage{mdframed} 
%   Frame box around a region

\usepackage{csquotes}
%   Quotes in the language is used in the text

\usepackage{xspace} 
%   Commands not to eat spaces

%\usepackage{mhchem} 
%   Chemical symbols

% \usepackage[numbers, square, merge, sort&compress]{natbib} 
%   Improved bibliography styles

\usepackage{notoccite}
%   Avoid citations at TOC   

\usepackage{tablefootnote}
%   Footnote at table environment 



%===============================================
%---Fonts---

%\usepackage[sc, osf]{mathpazo} 
%   Palatino mathematical font

%\usepackage{cmbright}

%\usepackage{avant} 
%   Avant Garde font

%\usepackage{helvet} 
%   Helvetica font

%\usepackage{eulervm} 
%   Euler virtual math fonts

\usepackage{times} 
%   Adobe Times Roman font

%\usepackage{mathptmx} 
%   Times mathematical font

%\usepackage{fourier} 
%   Utopia fonts

\usepackage{lettrine} 
%   Old german fonts

\usepackage{aurical} 
%   Lukas P fonts

%\usepackage{ascii}
%   Ascii fonts

\usepackage{wasysym} 
%   WASY2 fonts (astronomical, phonetic, musical, etc)

\usepackage{calligra} 
%   Calligraphic font

\usepackage{suetterl} 
%   Schulschriften german font

%\usepackage{euscript} 
%   Euler script font

%\usepackage[annataritalic]{tengwarscript} 
%   Tengwar (Elvish - The Lord of the Rings) fonts

%\usepackage[psfonts]{hiero} 
%   Hieroglyphic fons

\usepackage{trajan}%{\trjnfamily ABCDEFGHIJKLMNOPQRSTUVWXYZ} 
%   Old Roman font

%\usepackage{CJKutf8} 
%   Chinese-Japanese-Korean Typesetting

%\usepackage{kotex} 
%   Korean typesetting

%\usepackage{pxfonts}

%\usepackage{palatino}

%\usepackage{euler}

\usepackage{bbold}
%   "A geometric sans serif blackboard bold font"



%===============================================
%---Miscellaneous---

\usepackage{accents} 
%   Creation of accents and placement of scripts

\usepackage{chronology}
%   Createsa timeline   

\usepackage{cancel}
%   Adds cancel to math environment

\usepackage{fourier-orns}
%   Adds symbols

\usepackage[most]{tcolorbox}
%   Colored boxes
\newtcbtheorem{theo}{Theorem}{}{theorem}
%   Theorem inside colored boxes

\usepackage{empheq}
%   

\usepackage{tasks}
%   New environment tasks, a new list

\usepackage{blkarray}
%   New environment "Blockarray"

\usepackage{lscape}
%   

\usepackage{pdflscape}
%   Rotate the page horizontally
%   (LaTeX e pdfLaTeX, respectively)

\usepackage{lastpage}
%   Creates the "Last Page"

\usepackage{blindtext}
\usepackage{lipsum}
%    Adds the Lorem ipsum

%\usepackage{etoolbox} %Use carefully!

\usepackage{titling}
%   Title manipulation
\renewcommand\maketitlehooka{\null\mbox{}\vfill}
\renewcommand\maketitlehookd{\vfill\null}

\usepackage{tabu}
%   More useful tables

\usepackage{silence}
\WarningsOff[everypage]
%   Suppress warnings related to package everypage



%===============================================
%---Special Packages---

%%%%%%%%%%-HYPERREF-%%%%%%%%%%

\usepackage{hyperref}
%   Hypertext

\hypersetup{
	colorlinks = true,
	urlcolor   = blue,
	linkcolor  = blue,
	citecolor  = blue,
	%pdfborder  = {0,0,0},
	anchorcolor = blue, 
  filecolor   = blue, 
  menucolor   = blue, 
  linktocpage = true, 
  %bookmarks   = true,
  %pdfusetitle	
}
%   Setups the Hyperef Package

%%%%%%%%%%-FANCYHDR-%%%%%%%%%%

%\setlength{\textwidth}{15cm}

\usepackage{fancyhdr}                          
%   Decent headers and footers

\pagestyle{fancy}                          
%   For fancy pages - General document style

\renewcommand{\sectionmark}[1]{\markright{\thesection\ #1}}            

\fancyhf{}                                 %- clean configs/all fields

\fancyhead[LO,RE]{\bfseries\thepage}       %- external - page number
\fancyhead[RO,LE]{\nouppercase{\rightmark}}%- intrnl odd - chap/sect

%\lhead{\raisebox{0.1\height}{\includegraphics[height=25pt]{Images/Cover/LogoIFUSP(mncrm).png}}}

\renewcommand{\headrulewidth}{0.4pt}       %- hdr line width
\renewcommand{\footrulewidth}{0pt}         %- ft line width (null)
\addtolength{\headheight}{2.5pt}           %- gap for the hdr line
%   For fancy pages

\fancypagestyle{plain}{
   \fancyhead{}                            %- w/o hdrs at clean page
   \renewcommand{\headrulewidth}{0pt}      %- w/o hdr line
}
%   For plain pages

%\pagestyle{fancy}
%\fancyhead[L]{Sup. Esq.}
%\fancyhead[C]{Sup. Cent.}
%\fancyhead[R]{Sup. Dir.}

%\fancyfoot[L]{Inf. Esq.}
%\fancyfoot[C]{Page \thepage\ of \pageref{LastPage}}
%\fancyfoot[R]{Inf. Dir,}

%\renewcommand{\headrulewidth}{0.4pt}
%\renewcommand{\footrulewidth}{0.4pt}


%%%%%%%%%%-TOCLOFT-%%%%%%%%%%

%\usepackage{tocloft}
%\newcommand{\cftdot}{.}


%%%%%%%%%%-TITLESEC-%%%%%%%%%%

\usepackage{titlesec}
%   Section manipulation 

%\titleformat{\section}{\normalfont \Large \bfseries}{ \, \thesection}{2.3ex plus .2ex}{} 
% Text "Lecture" is what u wld like the section nmbr to dsply w/.
\titlespacing{\subsection}{2em}{*1}{*1}

\titleformat{\section}{\normalfont\Large\bfseries}{}{0pt}{}
%   Removes the numbers before section titles

%   Titlesec configs


%%%%%%%%%%-BACKGROUND-%%%%%%%%%%  

\usepackage[pages=some]{background}
%   Backgrounds

\backgroundsetup{
scale=1,
color=black,
opacity=0.25,
angle=0,
contents={
  \includegraphics[width=\paperwidth,height=\paperheight]{../Images/Cover/Background.png}
  }
}
%   Background config


%%%%%%%%%%-LAST PACKAGES-%%%%%%%%%%

%ALWAYS COMPILE W/ THOSE PACKAGES AT THE END AND AT THIS ORDER
%Unless the package in question explicitly says otherwise
%(U can remove the optional arguments if you wants)

\usepackage[english]{varioref} 
%   Better references

\usepackage[capitalize,english]{cleveref} 
%   Clever references

%-----------|--------------------------|-----------%
%           |                          |           %
%-----------|         COMMANDS         |-----------%
%           |__________________________|           %
%---%%%%%%%%%%%%%---% COMMANDS %---%%%%%%%%%%%%%---%

%===============================================
%---Math---

%%%%%%%%%%-SYMBOLS AND LETTERS-%%%%%%%%%%

\newcommand{\pa}{\partial}
\renewcommand{\S}{\mathcal{S}}      % Mathcal
\newcommand{\D}{\mathcal{D}}        % Mathcal
\newcommand{\F}{\mathcal{F}}        % Mathcal
\newcommand{\M}{\mathcal{M}}        % Mathcal
\renewcommand{\P}{\mathcal{P}}      % Mathcal
\renewcommand{\L}{\mathcal{L}}      % Mathcal
\renewcommand{\O}{\mathcal{O}}      % Mathcal
\renewcommand{\v}{\mathcal{v}}      % Mathcal
\renewcommand{\H}{\mathcal{H}}      % Mathcal
\newcommand{\I}{\mathcal{I}}      % Mathcal
\newcommand{\Z}{\mathbb{Z}}
\renewcommand{\Re}{\mathfrak{Re}}
\renewcommand{\Im}{\mathfrak{Im}}
\newcommand{\dg}{\dagger}
\newcommand{\be}{\begin{equation}}
\newcommand{\ee}{\end{equation}}
\newcommand{\keV}{{\rm keV}}        % Energy
\newcommand{\MeV}{{\rm MeV}}        % Energy
\newcommand{\GeV}{{\rm GeV}}        % Energy
\newcommand{\TeV}{{\rm TeV}}        % Energy
\newcommand{\pmns}{U_{\rm{PMNS}}}


%%%%%%%%%%-ELEMENTS-%%%%%%%%%%

\newcommand{\der}[2]{\frac{d#1}{d#2}}
\newcommand{\derf}[2]{\frac{\delta#1}{\delta#2}}
\newcommand{\dder}[2]{\frac{d^2#1}{d#2^2}}
\newcommand{\parc}[2]{\frac{\partial#1}{\partial #2}}
\newcommand{\pparc}[2]{\frac{\partial^2#1}{\partial #2^2}}
\newcommand{\crea}[2]{\hat{#1}^{\dag}_{#2}}
\newcommand{\des}[2]{\hat{#1}_{#2}}
\newcommand{\esp}[1]{\left[#1\right]}
\newcommand{\corc}[1]{\left(#1\right)}
\newcommand{\llav}[1]{\left\{#1\right\}}
\newcommand{\h}[1]{\hat{#1}}
\newcommand{\f}{f_{\nu_{j}}(\va{p},\va{P})}
\newcommand{\fd}{f^{*}_{\nu_{j}}(\va{p},\va{P})}
\newcommand{\sge}{\sigma_{p}^2}
\newcommand{\sprod}{\sigma_{p,P}^2}
\newcommand{\sd}{\sigma_{p,D}^2}
\newcommand{\al}[1]{\begin{align}\begin{aligned} #1 \end{aligned}\end{align}}
\newcommand{\inty}{\int^{\infty}_{-\infty}}


%%%%%%%%%%-OPERATORS AND FONTS-%%%%%%%%%%

\DeclareMathOperator{\defm}{:=}
\DeclareMathOperator{\sgn}{sgn}
\DeclareMathOperator{\tto}{\longrightarrow}
%\DeclareMathAlphabet{\mathpzc}{OT1}{pzc}{m}{it}
%\DeclareMathAlphabet{\mathbf}{U}{bf}{m}{n}
%\DeclareMathAlphabet{\mathfrak}{U}{frak}{m}{n}

%%%%%%%%%%-ENVIRONMENTS-%%%%%%%%%%

\newtheorem{definition}{\textcolor{black}{Definition}}
%   Definition command

\newtheorem{theorem}{\textcolor{black}{Theorem}}[section]
%   Theorem command

\newtheorem{lemma}[theorem]{\textcolor{black}{Lemma}}
%   Lemma command

\numberwithin{equation}{section}
%   Theorem number depends on the section

\renewcommand\qedsymbol{$\blacksquare$} 
%   Black proof square (QED symbol)

\newcommand{\m}{\text{-}}
\newcommand{\su}{\begin{bmatrix}
    0&1\\
    1&0
    \end{bmatrix}}
%   Special unitary group

\newtcolorbox{statement}[1]{
    enhanced,breakable,
    attach boxed title to top left={
    xshift=0.5cm,yshift= -3.5mm, },
    top=4mm,coltitle=white,
    beforeafter skip=\baselineskip,
    title=\textbf{#1}
}



%===============================================
%---Random---

%\renewcommand{\rmdefault}{futs}

\newcommand\exactfbox[1]{\fbox{\hskip1em#1\hskip1em}}
%   To box more than one line

\newcommand*\schulschrift{\fontfamily{wesu}\selectfont}

\newcommand{\ds}[1]{{\displaystyle #1 }}
\newcommand*{\pbar}[1]{\accentset{(-)}{#1}}
\newcommand{\CNB}{{\rm C}\nu{\rm B}}

\newcommand{\asection}[2]{
\setcounter{section}{#1}
\addtocounter{section}{-1}
    \section{#2}
}
%   Let u define the section number


%%%%%%%%%%%%%%%%%%%%% Margins %%%%%%%%%%%%%%%%%%%%%%%%

\setlength{\marginparwidth}{0pt}
\setlength{\marginparsep}{0pt}

%%%%%%%%%%%% Initial decorative letters %%%%%%%%%%%%

%\renewcommand{\LettrineFontHook}{\scshape}
%\setcounter{DefaultLines}{3}
%\renewcommand{\DefaultLoversize}{0}
%\AtBeginDocument{\setlength{\DefaultFindent}{0.5em}}
%\setlength{\DefaultNindent}{0pt}
%\renewcommand{\DefaultLraise}{0}

%%%%%%%%%%%%%%%%%%%%%%%%%%%%%%%%%%

%\setcounter{page}{1}

%\ChTitleVar{\large\rm\bf}
%\ChNameVar{\large\rm\bf}
%\ChNumVar{\large\bf}

%-----------|--------------------------|-----------%
%           |                          |           %
%-----------|         DOCUMENT         |-----------%
%           |__________________________|           %
%---%%%%%%%%%%%%%---% DOCUMENT %---%%%%%%%%%%%%%---%

%=======================================================
\graphicspath{{Images/}}
%   Path to images folder

\makeindex
%   Creates the index

\setlength{\parindent}{2em}
%   Paragraphs indentation size

\makeatletter
\def\@seccntformat#1{%
  \expandafter\ifx\csname c@#1\endcsname\c@subsection\else
  \csname the#1\endcsname\quad
  \fi}
\makeatother
%   wtf...

\AtBeginDocument{%
  \addtolength\abovedisplayskip{0.2\baselineskip}%
  \addtolength\belowdisplayskip{0.2\baselineskip}%
%  \addtolength\abovedisplayshortskip{-0.5\baselineskip}%
%  \addtolength\belowdisplayshortskip{-0.5\baselineskip}%
}

\addto\captionsenglish{\renewcommand*\contentsname{Summary}}
%   Change the Summary name



%-----------|----------------------|-----------%
%           |                      |           %
%-----------|         TIPS         |-----------%
%           |______________________|           %
%---%%%%%%%%%%%%%---% TIPS %---%%%%%%%%%%%%%---%

%=======================================================

%   To box more than one line:
%   Use like that:
%       \begin{empheq}[box=\exactfbox]{align*}
%           x &= 1 \\
%           y &= 1
%       \end{empheq}

%%%%%%%%%%%%%%%%%%%%%%%%%%%%%%

%   About summary:
%\setcounter{tocdepth}{1} % Show sections
%\setcounter{tocdepth}{2} % + subsections
%\setcounter{tocdepth}{3} % + subsubsections
%\setcounter{tocdepth}{4} % + paragraphs
%\setcounter{tocdepth}{5} % + subparagraphs

%%%%%%%%%%%%%%%%%%%%%%%%%%%%%%

%   Color palette
%       - Lavender: {135,99,219}

%%%%%%%%%%%%%%%%%%%%%%%%%%%%%

%   Wrap Figure
%       \begin{wrapfigure}{l}{0.2\textwidth}
%           \begin{center}
%           \includegraphics[width=0.2\textwidth]{Images/diagram-20220813.png}
%           \end{center}
%           \caption{Birds}
%           \label{fig:a}
%       \end{wrapfigure}

%%%%%%%%%%%%%%%%%%%%%%%%%%%%%

%\setcounter{secnumdepth}{0} - remove section numbering